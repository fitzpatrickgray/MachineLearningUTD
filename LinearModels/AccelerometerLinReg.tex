% Options for packages loaded elsewhere
\PassOptionsToPackage{unicode}{hyperref}
\PassOptionsToPackage{hyphens}{url}
%
\documentclass[
]{article}
\usepackage{amsmath,amssymb}
\usepackage{lmodern}
\usepackage{iftex}
\ifPDFTeX
  \usepackage[T1]{fontenc}
  \usepackage[utf8]{inputenc}
  \usepackage{textcomp} % provide euro and other symbols
\else % if luatex or xetex
  \usepackage{unicode-math}
  \defaultfontfeatures{Scale=MatchLowercase}
  \defaultfontfeatures[\rmfamily]{Ligatures=TeX,Scale=1}
\fi
% Use upquote if available, for straight quotes in verbatim environments
\IfFileExists{upquote.sty}{\usepackage{upquote}}{}
\IfFileExists{microtype.sty}{% use microtype if available
  \usepackage[]{microtype}
  \UseMicrotypeSet[protrusion]{basicmath} % disable protrusion for tt fonts
}{}
\makeatletter
\@ifundefined{KOMAClassName}{% if non-KOMA class
  \IfFileExists{parskip.sty}{%
    \usepackage{parskip}
  }{% else
    \setlength{\parindent}{0pt}
    \setlength{\parskip}{6pt plus 2pt minus 1pt}}
}{% if KOMA class
  \KOMAoptions{parskip=half}}
\makeatother
\usepackage{xcolor}
\usepackage[margin=1in]{geometry}
\usepackage{color}
\usepackage{fancyvrb}
\newcommand{\VerbBar}{|}
\newcommand{\VERB}{\Verb[commandchars=\\\{\}]}
\DefineVerbatimEnvironment{Highlighting}{Verbatim}{commandchars=\\\{\}}
% Add ',fontsize=\small' for more characters per line
\usepackage{framed}
\definecolor{shadecolor}{RGB}{248,248,248}
\newenvironment{Shaded}{\begin{snugshade}}{\end{snugshade}}
\newcommand{\AlertTok}[1]{\textcolor[rgb]{0.94,0.16,0.16}{#1}}
\newcommand{\AnnotationTok}[1]{\textcolor[rgb]{0.56,0.35,0.01}{\textbf{\textit{#1}}}}
\newcommand{\AttributeTok}[1]{\textcolor[rgb]{0.77,0.63,0.00}{#1}}
\newcommand{\BaseNTok}[1]{\textcolor[rgb]{0.00,0.00,0.81}{#1}}
\newcommand{\BuiltInTok}[1]{#1}
\newcommand{\CharTok}[1]{\textcolor[rgb]{0.31,0.60,0.02}{#1}}
\newcommand{\CommentTok}[1]{\textcolor[rgb]{0.56,0.35,0.01}{\textit{#1}}}
\newcommand{\CommentVarTok}[1]{\textcolor[rgb]{0.56,0.35,0.01}{\textbf{\textit{#1}}}}
\newcommand{\ConstantTok}[1]{\textcolor[rgb]{0.00,0.00,0.00}{#1}}
\newcommand{\ControlFlowTok}[1]{\textcolor[rgb]{0.13,0.29,0.53}{\textbf{#1}}}
\newcommand{\DataTypeTok}[1]{\textcolor[rgb]{0.13,0.29,0.53}{#1}}
\newcommand{\DecValTok}[1]{\textcolor[rgb]{0.00,0.00,0.81}{#1}}
\newcommand{\DocumentationTok}[1]{\textcolor[rgb]{0.56,0.35,0.01}{\textbf{\textit{#1}}}}
\newcommand{\ErrorTok}[1]{\textcolor[rgb]{0.64,0.00,0.00}{\textbf{#1}}}
\newcommand{\ExtensionTok}[1]{#1}
\newcommand{\FloatTok}[1]{\textcolor[rgb]{0.00,0.00,0.81}{#1}}
\newcommand{\FunctionTok}[1]{\textcolor[rgb]{0.00,0.00,0.00}{#1}}
\newcommand{\ImportTok}[1]{#1}
\newcommand{\InformationTok}[1]{\textcolor[rgb]{0.56,0.35,0.01}{\textbf{\textit{#1}}}}
\newcommand{\KeywordTok}[1]{\textcolor[rgb]{0.13,0.29,0.53}{\textbf{#1}}}
\newcommand{\NormalTok}[1]{#1}
\newcommand{\OperatorTok}[1]{\textcolor[rgb]{0.81,0.36,0.00}{\textbf{#1}}}
\newcommand{\OtherTok}[1]{\textcolor[rgb]{0.56,0.35,0.01}{#1}}
\newcommand{\PreprocessorTok}[1]{\textcolor[rgb]{0.56,0.35,0.01}{\textit{#1}}}
\newcommand{\RegionMarkerTok}[1]{#1}
\newcommand{\SpecialCharTok}[1]{\textcolor[rgb]{0.00,0.00,0.00}{#1}}
\newcommand{\SpecialStringTok}[1]{\textcolor[rgb]{0.31,0.60,0.02}{#1}}
\newcommand{\StringTok}[1]{\textcolor[rgb]{0.31,0.60,0.02}{#1}}
\newcommand{\VariableTok}[1]{\textcolor[rgb]{0.00,0.00,0.00}{#1}}
\newcommand{\VerbatimStringTok}[1]{\textcolor[rgb]{0.31,0.60,0.02}{#1}}
\newcommand{\WarningTok}[1]{\textcolor[rgb]{0.56,0.35,0.01}{\textbf{\textit{#1}}}}
\usepackage{graphicx}
\makeatletter
\def\maxwidth{\ifdim\Gin@nat@width>\linewidth\linewidth\else\Gin@nat@width\fi}
\def\maxheight{\ifdim\Gin@nat@height>\textheight\textheight\else\Gin@nat@height\fi}
\makeatother
% Scale images if necessary, so that they will not overflow the page
% margins by default, and it is still possible to overwrite the defaults
% using explicit options in \includegraphics[width, height, ...]{}
\setkeys{Gin}{width=\maxwidth,height=\maxheight,keepaspectratio}
% Set default figure placement to htbp
\makeatletter
\def\fps@figure{htbp}
\makeatother
\setlength{\emergencystretch}{3em} % prevent overfull lines
\providecommand{\tightlist}{%
  \setlength{\itemsep}{0pt}\setlength{\parskip}{0pt}}
\setcounter{secnumdepth}{-\maxdimen} % remove section numbering
\ifLuaTeX
  \usepackage{selnolig}  % disable illegal ligatures
\fi
\IfFileExists{bookmark.sty}{\usepackage{bookmark}}{\usepackage{hyperref}}
\IfFileExists{xurl.sty}{\usepackage{xurl}}{} % add URL line breaks if available
\urlstyle{same} % disable monospaced font for URLs
\hypersetup{
  pdftitle={Accelerometer Linear Regression},
  pdfauthor={Spencer Gray \& Michael Stinnett},
  hidelinks,
  pdfcreator={LaTeX via pandoc}}

\title{Accelerometer Linear Regression}
\author{Spencer Gray \& Michael Stinnett}
\date{2022-09-25}

\begin{document}
\maketitle

\begin{Shaded}
\begin{Highlighting}[]
\FunctionTok{library}\NormalTok{(ggplot2)}
\end{Highlighting}
\end{Shaded}

\hypertarget{dataset-source-httpsarchive.ics.uci.edumldatasetsaccelerometer}{%
\subsection{\texorpdfstring{Dataset Source:
\url{https://archive.ics.uci.edu/ml/datasets/Accelerometer\#}}{Dataset Source: https://archive.ics.uci.edu/ml/datasets/Accelerometer\#}}\label{dataset-source-httpsarchive.ics.uci.edumldatasetsaccelerometer}}

This dataset is a recording of movement in the x y and z directions in
response to variously placed weights on a fan. There are 153,000
recordings with 5 variables. There are 3 different configurations
(Wconfigid) which determines the arrangement of weights on the fan
blade. Pctid represents the power applied to the motor. In summary,
there are 3 setups each slowly increasing the motor power and recording
the disturbance.

Red is configuration 1\\
Blue is configuration 2\\
Green is configuration 3

\begin{figure}
\centering
\includegraphics[width=0.5\textwidth,height=\textheight]{fan.webp}
\caption{Fan Configuration, Credit:
\url{https://www.mdpi.com/1424-8220/19/19/4342/htm}}
\end{figure}

Predicting variations in Z movement based off of power going to the
motor (pctid).

\begin{Shaded}
\begin{Highlighting}[]
\NormalTok{accelerometer }\OtherTok{\textless{}{-}} \FunctionTok{read.csv}\NormalTok{(}\StringTok{"accelerometer.csv"}\NormalTok{)}
\FunctionTok{dim}\NormalTok{(accelerometer)}
\end{Highlighting}
\end{Shaded}

\begin{verbatim}
## [1] 153000      5
\end{verbatim}

\begin{Shaded}
\begin{Highlighting}[]
\FunctionTok{head}\NormalTok{(accelerometer)}
\end{Highlighting}
\end{Shaded}

\begin{verbatim}
##   wconfid pctid     x      y      z
## 1       1    20 1.004  0.090 -0.125
## 2       1    20 1.004 -0.043 -0.125
## 3       1    20 0.969  0.090 -0.121
## 4       1    20 0.973 -0.012 -0.137
## 5       1    20 1.000 -0.016 -0.121
## 6       1    20 0.961  0.082 -0.121
\end{verbatim}

\begin{Shaded}
\begin{Highlighting}[]
\NormalTok{mean1 }\OtherTok{\textless{}{-}} \FunctionTok{mean}\NormalTok{(accelerometer}\SpecialCharTok{$}\NormalTok{z)}
\NormalTok{med1 }\OtherTok{\textless{}{-}} \FunctionTok{median}\NormalTok{(accelerometer}\SpecialCharTok{$}\NormalTok{z)}
\NormalTok{range1 }\OtherTok{\textless{}{-}} \FunctionTok{range}\NormalTok{(accelerometer}\SpecialCharTok{$}\NormalTok{z)}
\NormalTok{var1 }\OtherTok{\textless{}{-}} \FunctionTok{var}\NormalTok{(accelerometer}\SpecialCharTok{$}\NormalTok{z)}
\NormalTok{sd1 }\OtherTok{\textless{}{-}} \FunctionTok{sd}\NormalTok{(accelerometer}\SpecialCharTok{$}\NormalTok{z)}

\FunctionTok{print}\NormalTok{(}\FunctionTok{paste}\NormalTok{(}\StringTok{\textquotesingle{}mean: \textquotesingle{}}\NormalTok{, mean1))}
\end{Highlighting}
\end{Shaded}

\begin{verbatim}
## [1] "mean:  -0.117769163398693"
\end{verbatim}

\begin{Shaded}
\begin{Highlighting}[]
\FunctionTok{print}\NormalTok{(}\FunctionTok{paste}\NormalTok{(}\StringTok{\textquotesingle{}median: \textquotesingle{}}\NormalTok{, med1))}
\end{Highlighting}
\end{Shaded}

\begin{verbatim}
## [1] "median:  -0.125"
\end{verbatim}

\begin{Shaded}
\begin{Highlighting}[]
\FunctionTok{print}\NormalTok{(}\FunctionTok{paste}\NormalTok{(}\StringTok{\textquotesingle{}range: \textquotesingle{}}\NormalTok{, range1))}
\end{Highlighting}
\end{Shaded}

\begin{verbatim}
## [1] "range:  -5.867" "range:  6.086"
\end{verbatim}

\begin{Shaded}
\begin{Highlighting}[]
\FunctionTok{print}\NormalTok{(}\FunctionTok{paste}\NormalTok{(}\StringTok{\textquotesingle{}var: \textquotesingle{}}\NormalTok{, var1))}
\end{Highlighting}
\end{Shaded}

\begin{verbatim}
## [1] "var:  0.267297088538572"
\end{verbatim}

\begin{Shaded}
\begin{Highlighting}[]
\FunctionTok{print}\NormalTok{(}\FunctionTok{paste}\NormalTok{(}\StringTok{\textquotesingle{}sd: \textquotesingle{}}\NormalTok{, sd1))}
\end{Highlighting}
\end{Shaded}

\begin{verbatim}
## [1] "sd:  0.517007822511973"
\end{verbatim}

\begin{Shaded}
\begin{Highlighting}[]
\FunctionTok{par}\NormalTok{(}\AttributeTok{mfrow=}\FunctionTok{c}\NormalTok{(}\DecValTok{1}\NormalTok{,}\DecValTok{3}\NormalTok{))}
\FunctionTok{with}\NormalTok{(accelerometer[accelerometer}\SpecialCharTok{$}\NormalTok{wconfid}\SpecialCharTok{\textless{}}\DecValTok{2}\NormalTok{,], }\FunctionTok{plot}\NormalTok{(pctid, z, }\AttributeTok{xlab=}\StringTok{"Motor Power"}\NormalTok{, }\AttributeTok{main=}\StringTok{"Red Configuration"}\NormalTok{))}
\FunctionTok{with}\NormalTok{(accelerometer[accelerometer}\SpecialCharTok{$}\NormalTok{wconfid}\SpecialCharTok{\textgreater{}}\DecValTok{1} \SpecialCharTok{\&}\NormalTok{ accelerometer}\SpecialCharTok{$}\NormalTok{wconfid }\SpecialCharTok{\textless{}} \DecValTok{3}\NormalTok{,], }\FunctionTok{plot}\NormalTok{(pctid, z,}\AttributeTok{xlab=}\StringTok{"Motor Power"}\NormalTok{, }\AttributeTok{main=}\StringTok{"Blue Configuration"}\NormalTok{))}
\FunctionTok{with}\NormalTok{(accelerometer[accelerometer}\SpecialCharTok{$}\NormalTok{wconfid}\SpecialCharTok{\textgreater{}}\DecValTok{2}\NormalTok{,], }\FunctionTok{plot}\NormalTok{(pctid, z,}\AttributeTok{xlab=}\StringTok{"Motor Power"}\NormalTok{, }\AttributeTok{main=}\StringTok{"Green Configuration"}\NormalTok{))}
\end{Highlighting}
\end{Shaded}

\includegraphics{AccelerometerLinReg_files/figure-latex/unnamed-chunk-2-1.pdf}

\hypertarget{linear-regression-model}{%
\subsection{Linear Regression Model}\label{linear-regression-model}}

\begin{Shaded}
\begin{Highlighting}[]
\FunctionTok{set.seed}\NormalTok{(}\DecValTok{1234}\NormalTok{)}
\NormalTok{i }\OtherTok{\textless{}{-}} \FunctionTok{sample}\NormalTok{(}\DecValTok{1}\SpecialCharTok{:}\FunctionTok{nrow}\NormalTok{(accelerometer), }\FunctionTok{nrow}\NormalTok{(accelerometer)}\SpecialCharTok{*}\FloatTok{0.8}\NormalTok{, }\AttributeTok{replace =} \ConstantTok{FALSE}\NormalTok{)}
\NormalTok{train }\OtherTok{\textless{}{-}}\NormalTok{ accelerometer[i,]}
\NormalTok{test }\OtherTok{\textless{}{-}}\NormalTok{ accelerometer[}\SpecialCharTok{{-}}\NormalTok{i,]}
\end{Highlighting}
\end{Shaded}

\hypertarget{statistical-analysis-of-linear-model-1-predictor}{%
\subsubsection{Statistical Analysis of Linear Model (1
predictor)}\label{statistical-analysis-of-linear-model-1-predictor}}

The relationship in the Residuals vs Fitted plot appears to show a
`linear' relationship. Moreso a line centered and parallel with the
x-axis. However this can be caused by a symmetrical oscillation centered
around the x-axis. We will most likely need more predictors and tweaks
to provide an accurate approximation.\\
The error (residuals) in the Q-Q plot appear to deviate strongly, follow
the line, and then deviate again. This could be caused by the three
separate configurations producing higher or lower variance. This is
further revealed in the scale-location plot revealing that our z-axis
variance increases as we increase our motor power. Residuals vs Leverage
plot does not seem to share any substantial information on strong
outliers

\begin{Shaded}
\begin{Highlighting}[]
\NormalTok{lm1 }\OtherTok{\textless{}{-}} \FunctionTok{lm}\NormalTok{(z }\SpecialCharTok{\textasciitilde{}}\NormalTok{ pctid, }\AttributeTok{data=}\NormalTok{train)}
\FunctionTok{summary}\NormalTok{(lm1)}
\end{Highlighting}
\end{Shaded}

\begin{verbatim}
## 
## Call:
## lm(formula = z ~ pctid, data = train)
## 
## Residuals:
##     Min      1Q  Median      3Q     Max 
## -5.7570 -0.0543 -0.0018  0.0515  4.9039 
## 
## Coefficients:
##               Estimate Std. Error t value Pr(>|t|)    
## (Intercept) -1.289e-01  3.911e-03 -32.968  < 2e-16 ***
## pctid        1.898e-04  6.038e-05   3.143  0.00167 ** 
## ---
## Signif. codes:  0 '***' 0.001 '**' 0.01 '*' 0.05 '.' 0.1 ' ' 1
## 
## Residual standard error: 0.5175 on 122398 degrees of freedom
## Multiple R-squared:  8.072e-05,  Adjusted R-squared:  7.255e-05 
## F-statistic: 9.881 on 1 and 122398 DF,  p-value: 0.00167
\end{verbatim}

\begin{Shaded}
\begin{Highlighting}[]
\FunctionTok{par}\NormalTok{(}\AttributeTok{mfrow=}\FunctionTok{c}\NormalTok{(}\DecValTok{2}\NormalTok{,}\DecValTok{2}\NormalTok{))}
\FunctionTok{plot}\NormalTok{(lm1)}
\end{Highlighting}
\end{Shaded}

\includegraphics{AccelerometerLinReg_files/figure-latex/unnamed-chunk-4-1.pdf}

\hypertarget{statistical-analysis-of-linear-model-2-predictors}{%
\subsubsection{Statistical Analysis of Linear Model (2
predictors)}\label{statistical-analysis-of-linear-model-2-predictors}}

\begin{Shaded}
\begin{Highlighting}[]
\NormalTok{lm2 }\OtherTok{\textless{}{-}} \FunctionTok{lm}\NormalTok{(z }\SpecialCharTok{\textasciitilde{}}\NormalTok{ pctid}\SpecialCharTok{+}\NormalTok{wconfid, }\AttributeTok{data=}\NormalTok{train)}
\FunctionTok{summary}\NormalTok{(lm2)}
\end{Highlighting}
\end{Shaded}

\begin{verbatim}
## 
## Call:
## lm(formula = z ~ pctid + wconfid, data = train)
## 
## Residuals:
##     Min      1Q  Median      3Q     Max 
## -5.7560 -0.0544 -0.0018  0.0516  4.9039 
## 
## Coefficients:
##               Estimate Std. Error t value Pr(>|t|)    
## (Intercept) -1.310e-01  5.333e-03 -24.557  < 2e-16 ***
## pctid        1.898e-04  6.038e-05   3.144  0.00167 ** 
## wconfid      1.016e-03  1.811e-03   0.561  0.57493    
## ---
## Signif. codes:  0 '***' 0.001 '**' 0.01 '*' 0.05 '.' 0.1 ' ' 1
## 
## Residual standard error: 0.5175 on 122397 degrees of freedom
## Multiple R-squared:  8.329e-05,  Adjusted R-squared:  6.695e-05 
## F-statistic: 5.098 on 2 and 122397 DF,  p-value: 0.006112
\end{verbatim}

\begin{Shaded}
\begin{Highlighting}[]
\FunctionTok{par}\NormalTok{(}\AttributeTok{mfrow=}\FunctionTok{c}\NormalTok{(}\DecValTok{2}\NormalTok{,}\DecValTok{2}\NormalTok{))}
\FunctionTok{plot}\NormalTok{(lm2)}
\end{Highlighting}
\end{Shaded}

\includegraphics{AccelerometerLinReg_files/figure-latex/unnamed-chunk-5-1.pdf}

\hypertarget{statistical-analysis-using-other-methods}{%
\subsubsection{Statistical Analysis using Other
Methods}\label{statistical-analysis-using-other-methods}}

To get a more accurate estimate I am selecting one sample of tests, fan
configuration 1. Then taking the absolute value of the z movements (to
remove symmetry causing net zero slope) and doing a polynomial
regression.

\begin{Shaded}
\begin{Highlighting}[]
\NormalTok{lm3 }\OtherTok{\textless{}{-}} \FunctionTok{with}\NormalTok{(train[train}\SpecialCharTok{$}\NormalTok{wconfid}\SpecialCharTok{\textless{}}\DecValTok{2}\NormalTok{,], }\FunctionTok{lm}\NormalTok{(}\FunctionTok{sqrt}\NormalTok{(}\FunctionTok{abs}\NormalTok{(z)) }\SpecialCharTok{\textasciitilde{}}\NormalTok{ pctid }\SpecialCharTok{+} \FunctionTok{I}\NormalTok{(pctid}\SpecialCharTok{\^{}}\DecValTok{2}\NormalTok{) }\SpecialCharTok{+} \FunctionTok{I}\NormalTok{(pctid}\SpecialCharTok{\^{}}\DecValTok{3}\NormalTok{) }\SpecialCharTok{+}\NormalTok{ wconfid, }\AttributeTok{data =}\NormalTok{ train))}
\FunctionTok{summary}\NormalTok{(lm3)}
\end{Highlighting}
\end{Shaded}

\begin{verbatim}
## 
## Call:
## lm(formula = sqrt(abs(z)) ~ pctid + I(pctid^2) + I(pctid^3) + 
##     wconfid, data = train)
## 
## Residuals:
##      Min       1Q   Median       3Q      Max 
## -0.88436 -0.11438 -0.00127  0.10228  1.53783 
## 
## Coefficients:
##               Estimate Std. Error t value Pr(>|t|)    
## (Intercept)  8.142e-01  9.455e-03   86.11   <2e-16 ***
## pctid       -1.730e-02  5.620e-04  -30.79   <2e-16 ***
## I(pctid^2)   2.873e-04  1.012e-05   28.39   <2e-16 ***
## I(pctid^3)  -9.760e-07  5.585e-08  -17.47   <2e-16 ***
## wconfid     -9.693e-02  7.748e-04 -125.10   <2e-16 ***
## ---
## Signif. codes:  0 '***' 0.001 '**' 0.01 '*' 0.05 '.' 0.1 ' ' 1
## 
## Residual standard error: 0.2214 on 122395 degrees of freedom
## Multiple R-squared:  0.3692, Adjusted R-squared:  0.3692 
## F-statistic: 1.791e+04 on 4 and 122395 DF,  p-value: < 2.2e-16
\end{verbatim}

\begin{Shaded}
\begin{Highlighting}[]
\FunctionTok{par}\NormalTok{(}\AttributeTok{mfrow=}\FunctionTok{c}\NormalTok{(}\DecValTok{2}\NormalTok{,}\DecValTok{2}\NormalTok{))}
\FunctionTok{plot}\NormalTok{(lm3)}
\end{Highlighting}
\end{Shaded}

\includegraphics{AccelerometerLinReg_files/figure-latex/unnamed-chunk-6-1.pdf}

As seen in both linear model 1 and linear model 2, they both got similar
results. A misrepresentation of the data and a prediction that doesn't
even come close to the test data. The statistical summary listed below
shows a covariance less than 0.01, but with linear model 3 we got a
covariance of 0.6. The first step was to separate the three different
categories. For model 3 I restricted the test and training data to fan
configuration 1. Then removed x-axis symmetry with abs() before using
sqrt() to reduce the exponential behavior. A polynomial fit with a
degree of 3 was used for best fit, compensating for any quadratic
behavior.

\hypertarget{statistical-summary}{%
\subsection{Statistical Summary}\label{statistical-summary}}

\begin{Shaded}
\begin{Highlighting}[]
\NormalTok{pred1 }\OtherTok{\textless{}{-}} \FunctionTok{predict}\NormalTok{(lm1, }\AttributeTok{newdata =}\NormalTok{ test)}
\NormalTok{cor1 }\OtherTok{\textless{}{-}} \FunctionTok{cor}\NormalTok{(pred1, test}\SpecialCharTok{$}\NormalTok{z)}
\NormalTok{mse1 }\OtherTok{\textless{}{-}} \FunctionTok{mean}\NormalTok{((pred1 }\SpecialCharTok{{-}}\NormalTok{ test}\SpecialCharTok{$}\NormalTok{z)}\SpecialCharTok{\^{}}\DecValTok{2}\NormalTok{)}
\NormalTok{rmse1 }\OtherTok{\textless{}{-}} \FunctionTok{sqrt}\NormalTok{(mse1)}

\NormalTok{pred2 }\OtherTok{\textless{}{-}} \FunctionTok{predict}\NormalTok{(lm2, }\AttributeTok{newdata =}\NormalTok{ test)}
\NormalTok{cor2 }\OtherTok{\textless{}{-}} \FunctionTok{cor}\NormalTok{(pred2, test}\SpecialCharTok{$}\NormalTok{z)}
\NormalTok{mse2 }\OtherTok{\textless{}{-}} \FunctionTok{mean}\NormalTok{((pred2 }\SpecialCharTok{{-}}\NormalTok{ test}\SpecialCharTok{$}\NormalTok{z)}\SpecialCharTok{\^{}}\DecValTok{2}\NormalTok{)}
\NormalTok{rmse2 }\OtherTok{\textless{}{-}} \FunctionTok{sqrt}\NormalTok{(mse2)}

\NormalTok{pred3 }\OtherTok{\textless{}{-}} \FunctionTok{with}\NormalTok{(test[test}\SpecialCharTok{$}\NormalTok{wconfid}\SpecialCharTok{\textless{}}\DecValTok{2}\NormalTok{,], }\FunctionTok{predict}\NormalTok{(lm3, }\AttributeTok{newdata =}\NormalTok{ test))}
\NormalTok{cor3 }\OtherTok{\textless{}{-}} \FunctionTok{cor}\NormalTok{(pred3, }\FunctionTok{sqrt}\NormalTok{(}\FunctionTok{abs}\NormalTok{(test}\SpecialCharTok{$}\NormalTok{z)))}
\NormalTok{mse3 }\OtherTok{\textless{}{-}} \FunctionTok{mean}\NormalTok{((pred3 }\SpecialCharTok{{-}} \FunctionTok{sqrt}\NormalTok{(}\FunctionTok{abs}\NormalTok{(test}\SpecialCharTok{$}\NormalTok{z)))}\SpecialCharTok{\^{}}\DecValTok{2}\NormalTok{)}
\NormalTok{rmse3 }\OtherTok{\textless{}{-}} \FunctionTok{sqrt}\NormalTok{(mse3)}

\FunctionTok{print}\NormalTok{(}\FunctionTok{paste}\NormalTok{(}\StringTok{\textquotesingle{}correlation: \textquotesingle{}}\NormalTok{, cor1, cor2, cor3))}
\end{Highlighting}
\end{Shaded}

\begin{verbatim}
## [1] "correlation:  0.0088989796081442 0.00959082756295296 0.611563022552147"
\end{verbatim}

\begin{Shaded}
\begin{Highlighting}[]
\FunctionTok{print}\NormalTok{(}\FunctionTok{paste}\NormalTok{(}\StringTok{\textquotesingle{}mse: \textquotesingle{}}\NormalTok{, mse1, mse2, mse3))}
\end{Highlighting}
\end{Shaded}

\begin{verbatim}
## [1] "mse:  0.265270527756122 0.26526716143052 0.0486342589125535"
\end{verbatim}

\begin{Shaded}
\begin{Highlighting}[]
\FunctionTok{print}\NormalTok{(}\FunctionTok{paste}\NormalTok{(}\StringTok{\textquotesingle{}rmse: \textquotesingle{}}\NormalTok{, rmse1, rmse2, rmse3))}
\end{Highlighting}
\end{Shaded}

\begin{verbatim}
## [1] "rmse:  0.515044199808252 0.515040931801076 0.220531763953752"
\end{verbatim}

\hypertarget{summary}{%
\subsubsection{Summary}\label{summary}}

The first two models included all of the different fan configurations,
resulting in inconsistent results. The biggest factor that separated the
third model is to categorize data properly before training a model.

\end{document}
